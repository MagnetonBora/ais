\documentclass[a4paper,14pt, openany, twoside, draft]{extbook} % computer modern font calls
% При завершении (верстке) заменить draft на final!!!!
\usepackage[final]{graphicx}
% \graphicspath{{./pics/}{./paropt/pics/}} % Местонахождение картинок
\usepackage[usenames]{xcolor}
% \usepackage[final]{minted} % Удобный пакет для представления программного кода, требует внешнюю программу и специальные параметры запуска. В отличие от пакета listings поддерживаются русские буквы и utf-8.
%\usemintedstyle{emacs} % Разные стили цвета текста
%\usemintedstyle{tango}
%\usemintedstyle{trac} % better (more bold faces}
%\usemintedstyle{manni} % pastel colors
%\usemintedstyle{bw}    % Черно-белый стиль, следует использовать при верстке печатных документов

%\setminted{breaklines=true,fontsize=\small,funcnamehighlighting=true,python3=true} % Настройка параметров пакета minted

%\newminted{prolog}
%\protect\newminted[proexp]{prolog}{style=bw} % Создание специального окружения для кода языка Prolog.  Окружение позволяет не расцвечивать примеры-запросы.

% Main style definition
% ---------------------
% ISU standard handbook.
%\usepackage[times,fancybot,firamono]{subook} % looks less good than inconsolata
\usepackage[times,fancybot,inconsolata]{subook} % Собственно загрузка стиля.
%\usepackage[times]{subook}

% Заменить в коде кой-что на математические формулы
% http://www.tug.org/texlive/Contents/live/texmf-dist/doc/latex/base/alltt.pdf

% Требования к оформлению Типографии ИГУ
% http://lawinstitut.ru/ru/about/services/izdatelstvo/trebovaniya.html
%

% ISU standard monograph (less restrictive and more artistic)
% Из этого примера можно взять настройки стиля для нашего стиля ;-).
% mag не надо использовать! Он тут чисто исторически остался.
% \usepackage[monograph,mag,times,smalltitles,fancybot,listbib,ptfonts,microtyping]{subook}

% Some artistism for monograph
% ----------------------------
%\makeatletter{}
%\renewcommand\su@chapter@font{\sffamily\sfcpshape\bfseries}
%\renewcommand\su@chapter@font@size{\LARGE}
%\makeatother{}
%\floatname{algorithm}{Процедура}
%\renewcommand{\listalgorithmname}{Список процедур}
%\renewcommand\cftsecnumwidth{5ex}
%\tolerance=5000
%\renewcommand{\chaptername}{Глава}
%\usepackage[final]{hyperref}

\definecolor{mygreen}{rgb}{0,0.6,0} % Цвета для отображения комментариев.
\definecolor{mygray}{rgb}{0.5,0.5,0.5}
\definecolor{mymauve}{rgb}{0.58,0,0.82}


\usepackage{tikz}  % Очень мощный пакет для рисования, но и одновременно очень сложный.
\usetikzlibrary{arrows,arrows.meta,shapes} % Библиотеки для пакета.
\usetikzlibrary{shadows}
\newcommand*\keystroke[1]{% Команда используется для отображения нажатий клавиш.
  \tikz[baseline=(key.base)]
    \node[%
      draw,
      fill=white,
      drop shadow={shadow xshift=0.25ex,shadow yshift=-0.25ex,fill=black,opacity=0.75},
      rectangle,
      rounded corners=4pt,
      inner sep=1pt,
      line width=0.7pt,
      font=\footnotesize\sffamily
    ](key) {~#1~\strut}%
  ;%
}

\long\def\rem#1{} % Используется для комментирования больших кусков текста.
\def\emphbib#1{#1} % Заглушка для оформления стилей заголовков книг в списке литературы
%\newenvironment{questions}{\subsubsection*{Вопросы для самопроверки}\begin{enumerate}\itemsep0pt minus 0.3pt\parskip0pt plus 0.3pt}{\end{enumerate}}

%\newtheorem{example}{Пример}[chapter]
\hypersetup{
    bookmarks=true,         % show bookmarks bar?
    unicode=true,           % non-Latin characters in Acrobat’s bookmarks
    pdftoolbar=true,        % show Acrobat’s toolbar?
    pdfmenubar=true,        % show Acrobat’s menu?
    pdffitwindow=false,     % window fit to page when opened
    pdfstartview={FitH},    % fits the width of the page to the window
    pdftitle={Название работы},    % title
    pdfauthor={Автор},     % author
    pdfsubject={О чем идет речь в работе},   % subject of the document
    pdfcreator={EMACS-24.5:AuCTeX},   % creator of the document
    pdfproducer={LuaLaTeX}, % producer of the document
    pdfkeywords={Ключевое слово 1} {Ключевое слово 2} {Ключевое слово 3}, % list of keywords
    pdfnewwindow=true,      % links in new window
    colorlinks=true,       % false: boxed links; true: colored links
    linkcolor=[rgb]{0 0.4 0.1},          % color of internal links (black)
    citecolor=blue,        % color of links to bibliography
    filecolor=black,      % color of file links
    urlcolor=[rgb]{0.3 0.0 0.3}           % color of external links
}

%\renewcommand{\headrulewidth}{1pt}

\clubpenalty=3000
\widowpenalty=3000
%\brokenpenalty=10000
%\floatingpenalty=10000

%% \setdefaultlanguage{russian}
%% \setmainlanguage{russian}
%% \setotherlanguage{english}

%\newenvironment{mygroup}{}{}

\renewcommand\baselinestretch{1.5} % Для диплома.

\definecolor{rclr}{rgb}{0.5,0.1,0.1}
\definecolor{eclr}{rgb}{0,0.5,0.5}
\colorlet{acolor}{blue}
\colorlet{rcolor}{red}
\definecolor{ncolor}{rgb}{0.5,0.5,0.1}
\newcommand{\aaa}[2][acolor]{\noindent\textcolor{eclr}% Использую для пометки места, где надо текста добавить
{+\ [}\textcolor{#1}{#2}\textcolor{eclr}{]}}
\newcommand{\rrr}[2][rcolor]{\noindent% Использую для пометки места, где текста надо убрать
\textcolor{eclr}{-\ [}\textcolor{#1}{#2}\textcolor{eclr}{]}}
\newcommand{\nnn}[2][ncolor]{\noindent% Использую для пометки места, где надо на что-то обратить внимание
\textcolor{eclr}{!\ [}\textcolor{#1}{#2}\textcolor{eclr}{]}}
%\newcommand{\goforth}[1]{$\,\hookrightarrow$\pageref{#1}} % Фича для рисования знака быстрого перехода через раздел, если нет его смысла читать.

% \begin{figure} % Шаблон для включения .pdf_latex - файлов, генерируемых редактором Inkscape
%   \centering
%   \def\svgwidth{\columnwidth}
%   \includesvg{image}
% \end{figure}

%Настройка ниже поджимает абзацы друг к другу, но визуально никак не заметно (пока).
\parskip=0pt plus 0.3pt
\begin{document}
% \itemsep3pt plus 0pt minus 3pt
% \widowpenalty=10000
% \clubpenalty=10000
% \renewcommand\sutitlefontface{\Large\ptsans\nwshape\bfseries}
% \theorembodyfont{\rmfamily}

%\renewcommand{\chaptername}{} % for ISU Handbooks
\renewcommand{\refname}{Список использованных источников} % ... also
\renewcommand{\bibname}{\refname}
\begin{titlepage}
\thispagestyle{empty}
%\aaa{Эта и следующая страница вставляется из .docx}
\begin{center}{\small{}
Министерство образования и науки
Российской Федерации \\
Федеральное государственное бюджетное образовательное\\
учреждение высшего профессионального образования\\
<<Иркутский государственный университет>> \\
Институт математики, экономики и информатики
%\\[2ex]
%    Учреждение Российской академии наук \\
%<<Институт динамики систем и теории управления \\
%Сибирского отделения РАН>>
}
\vfill
\hbox to \linewidth{\hfill\bfseries И.~С.~Петров\hfill}
 \vspace{2em}
{\large\bfseries РАЗРАБОТКА СУПЕР"=ПУПЕРНОЙ СИСТЕМЫ} % "= - это мелкий дефис, который не подавляет переносы в словах, т.е. LaTeX2e не переносит слова "красно-зеленый", он воспринимает оба слова и тире как одно неделимое слово, чтоб переносы работали надо вместо "-" писать "= (двойная кавычка и равно)
\vspace{2em}
{Дипломная работа}
\vfill
%\vfill
\vfill
 \textbf{Иркутск 2014}
\end{center}
\end{titlepage}

\newpage
\thispagestyle{empty}
\begin{center}
  \Large\bfseries Аннотация
\end{center}
Аннотация, 1 страница не более. Лучше 2 абзаца\,: первый~--- краткое описание задачи, второй~--- краткое описание результата.  Если не лень, то 3-й~--- краткое описание метода решения, 4-й~--- краткое описание примера приложения результата.

В дипломной работе рассмотрена задача \ldots{}

В результате проведенных исследований и разработки программного обеспечения получено\ldots{}

\clearpage
%\setcounter{page}{2}
\tableofcontents
\clearpage

\newpage
\chapter*{Введение}

\rrr{Перед тема как начать изложение структуры квалификационной работы, я прошу не копировать текст из данного шаблона один"=к"=одному.  Потом это будет выглядеть как"=то ненормально, если тексты дипломов будут выглядеть как сиамские близнецы. }

В этом разделе, Введении, необходимо описать \emph{проблемную область}, \emph{конкретную задачу} исследования или проектирования, \emph{объект} и \emph{предмет исследования}.  Затем необходимо перейти к \emph{обоснованию актуальности} решаемой в дипломной работе задаче.  Актуальность удобно излагать в самом конце подготовки текста.  Актуальность~--- едва ли не самая ответственная часть текста.  Если ей не уделить должного внимания, то могут возникнуть вопросы целесообразности работы~--- Нужно ли было проводить данную работу, тратить на нее свое время и время преподавателя.

Следующий важный момент Введения~--- описание \emph{цели} и \emph{задач} квалификационной работы.  Сначала ставится цель~--- предсказуемый результат исследования или разработки~---, затем список задач.  Предполагается, что цель достигнута, если все задачи из списка полностью решены.  Поэтому задачи, по своей сути,~--- это план реализации исследования, план построения решения.

В следующем примере задачи из дипломных работ перемешаны с задачами дипломных проектов. \ldots{} Для достижения указанной цели решены следующие задачи\,:
\begin{itemize}
\item Разработка модели объекта исследования\ldots{}; % Обращаю внимание на точку с запятой в конце данной строки!
\item Моделирование организационной структуры предприятия;
\item Разработка архитектуры системы\ldots{} ;
\item Создание информационной модели приложения;
\item Обоснование \ldots{} свойств модели объекта \nnn{(Здесь уже объект заменяется на его реальное название, например, АНПА, самолет, рост прибыли предприятия)};
\item \ldots{};
\item Реализация разработанных алгоритмов \ldots{} в виде программы на языке \ldots{};
\item Проведение тестирования разработанного программного комплекса на данных \ldots{}.
\end{itemize}

Задач должно быть 5-7, чтобы было легко воспринимаемо, а также задачи должны быть логически связанными.  Затем, практически всегда уделяется внимание трамбованиям к решению, поиском которого занимаемся.  Требования выражают возможности инженера или исследователя, его инструментарий, в конкретной ситуации.  Например, на предприятии предпочитают использовать продукты компании Microsoft.  Это обозначает, что инженер уде не может поставить ``любимый'' Linux на северную машину.  Другой пример~--- решение задачи оптимального управления должно быть получено аналитически.  То есть, требования как ограничения мешают ``нам жить'', но при этом приближают нас к ``естественной среде обитания''.  Требования во втором примере играют роль общей схемы решения задачи, что, иногда похоже на заготовленную подсказку со стороны заказчика.

\emph{Литературный обзор} помещается либо непосредственно перед актуальностью, либо сразу после нее, либо сразу после цели и задач.  Литературный обзор должен быть!  Не бывает таких задач, которыми хотя бы в малой мере никто не занимался.

\nnn{Далее дописываем, если не лень, структуру текста отчета (дипломной работы или пояснительной записки к дипломному проекту.}

В \hyperref[cha:first]{первой главе} рассматриваются теоретические основы решения задачи\ldots{} В \hyperref[sec:first:one]{п.~\ref{sec:first:one}} рассматривается\ldots{}, в п.\ldots{}  В \hyperref[cha:second]{Главе~\ref{cha:second}}\ldots.

В \hyperref[cha:conc]{Заключении} приведены основные результаты дипломной работы.  В \hyperref[cha:app]{Приложении} представлен исходный код модулей системы.

% $$
% \int\limits_1^1f(x)dx.
% $$


\chapter{Теоретические основы \ldots{}}
\label{cha:first}

\section{Определения}
\label{sec:first:one}

\section{Постановка задачи }

\chapter{Реализация и тестирование программной системы\ldots{}}

\chapter{Приложение программной системы в \ldots{}}

\chapter*{Заключение}

Заключение, обычно, оформляется в первую очередь, ведь кому как ни вам знать, что конкретно было сделано в дипломной работе.  Перечень проделанной работы вставляется в виде списка сюда, и уже отсюда в трансформированном виде переносится во введение.

Заключение начинается с фразы, характеризующей проделанную работу в общем, затем перечисляются решенные задачи.  Список задач в заключении, по идее, должен совпадать со списком из Введения.  В списка решенных задач не плохо смотрятся фразы, характеризующие проделанный этап, например
\begin{itemize}
\item[5.] При помощи метода ER"=диаграмм разработана логическая и физическая модели приложения.
\end{itemize}

Далее, в подбор, оценивается достижение цели.  В следующих двух абзацах оценивается качество решения задачи в общих терминах, выделяются важные моменты решения, как положительные, так и отрицательные.  В последнем абзаце уделяется внимание дальнейшему развитию проведенного исследования, разработанной технологии и программного обеспечения.


\nnn{В данном списке литературы приведены примеры оформления различных источников.  Там сейчас есть ошибки наверняка.}

%\listoffigures
%\addcontentsline{toc}{section}{Список иллюстраций}
%\listoftables
%\addcontentsline{toc}{section}{Список таблиц}
\begin{thebibliography}{99}\itemsep1pt \parskip 0pt plus 0.3pt
\bibitem{pontr} Понтрягин~Л.~С. Принцип максимума в оптимальном управлении. Изд.~2-е, стереотипное. М.\,: Едиториал УРСС,  2004.~--~64~с.
\bibitem{rzhddb} URL:\href{http://mapservis.ru/docs/tar_ruc_4.htm}{Тарифное руководство № 4. Книга 2. Часть 1. Алфавитный список железнодорожных станций.} [Электронный ресурс]\,{}: сайт. \url{http://mapservis.ru/docs/tar_ruc_4.htm} (дата обращения: 06.05.2015).
\bibitem{citycoords} \href{http://alextyurin.ru/?p=1037}{Географические координаты основных городов России}. [Электронный ресурс]\,{}: сайт. URL:\url{http://alextyurin.ru/?p=1037} (дата обращения: 06.05.2015).
\bibitem{Anderson} Андерсон~Р. \emphbib{Доказательство правильности программ}\,{}: пер. с англ.\,{}/ Р.~Андерсон. -- М.\,:\,Мир, 1982. -- 168~c.: ил.
\bibitem{Bratko} Братко~И. \emphbib{\href{http://royallib.ru/book/bratko_ivan/programmirovanie_na_yazike_prolog_dlya_iskusstvennogo_intellekta.html}{Программирование на языке ПРОЛОГ для искусственного интеллекта}}\,{}: пер. с англ.\,/ И.~Братко. -- М.\,:~Мир, 1990. -- 560~c.: ил.
\bibitem{Vass:2000} Васильев~С.~Н. \emphbib{\href{http://bookfi.org/book/616050}{Интеллектное управление динамическими системами}}\,{}/ С.~Н.~Васильев, А.~К.~Жерлов, Е.~А.~Федосов, Б.~Е.~Федунов. -- М.\,:~Физматлит, 2000. -- 352~с: ил.
\bibitem {AIDictionary} \emphbib{\href{http://aihandbook.intsys.org.ru/index.php/intro/ai-handbook}{Искусственный интеллект\,{}: в 3~кн.}}\,{}/ под ред. Э.~В. Попова. -- М.\,:~Радио и связь, 1990. -- 464 c.:\,{}ил.
\bibitem{Lauriere} Лорьер.~Ж.-Л.  \emphbib{\href{http://publ.lib.ru/ARCHIVES/L/LOR'ER_Jan_Lui/_Lor'er_J.L..html}{Системы искусственного интеллекта}\,{}: пер. с франц.}\,{}/ Ж.-Л. Лорьер. -- М.\,:~Мир, 1991. -- 568~с.: ил.
\bibitem{Malpas} Малпас~Дж. \emphbib{\href{http://padaread.com/?book=40731&pg=1}{Реляционный язык Пролог и его применение}}\,{}/ Дж.~Малпас. -- М.\,:~Наука, 1990. -- 464~с.
\bibitem{math_slov:88} \emphbib{\href{https://app.box.com/shared/793ukgvblxmj0hh6btw4}{Математический энциклопедический словарь}}\,{}/ гл.~ред. Ю.~В.~Прохоров. -- М.\,:~Сов.~энциклопедия, 1988. -- 847~c.
\bibitem{DDW} Непейвода~Н.~Н. \emphbib{\href{http://www.logic-books.info/taxonomy/term/215}{Прикладная логика\,{}: учеб. пособие}}\,{}/ Н.~Н.~Непейвода. -- 2-е изд. -- Новосибирск\,{}:~Изд-во Новосиб. ун-та, 2000. -- 521~c.: ил.
%\bibitem{DDWII} Непейвода~Н.~Н.  \emphbib{\href{http://philosophy.ru/library/logic_math/library/nepeivoda_prog.pdf}{Основания программирования}}\,{}/ Н.~Н.~Непейвода, И.~Н.~Скопин. -- Москва; Ижевск\,{}:~Институт компьютерных исследований, 2003 -- 880~c.: ил.
\bibitem {Russell} Рассел~С. \href{http://www.aiportal.ru/downloads/books/ai-modern-approach-2-edition-by-rassel-norvig.html}{Искусственный интеллект: современный подход}\,{}: пер. с англ.\,{}/ С.~Рассел, П.~Новриг. 2-е изд. -- М.\,:~Изд. дом <<Вильямс>>, 2006. -- 1408~c.: ил.
\bibitem{WIKI-DCG} \emphbib{\href{https://en.wikipedia.org/wiki/Definite_clause_grammar}{DC-грамматика}} [Электронный ресурс]\,{}// Wikipedia, The Free Encyclopedia\,{}: сайт. -- URL:\texttt{https://en.wikipedia.org/wiki/Definite\_clause\linebreak\_grammar}. (дата обращения: 28.11.2013).
\bibitem{GNUP} \emphbib{\href{http://www.gprolog.org/}{The GNU Prolog web site} [Электронный ресурс]\,{}: сайт}. URL:\url{http://www.gprolog.org/}. (дата обращения: 28.11.2013).
\bibitem{SWIP} \emphbib{\href{http://www.swi-prolog.org/}{SWI-Prolog's home} [Электронный ресурс]\,{}: сайт}. URL:\url{http://www.swi-prolog.org/}. (дата обращения: 28.11.2013).

\bibitem{pythondl} Welcome to Python.org.  [Электронный ресурс]\,{}: сайт. URL:\url{https://www.python.org/} (дата обращения: 11.01.2015).
\bibitem{pythondoc} \href{http://younglinux.info/sites/default/files/python_structured_programming.pdf}{Основы программирования на Python.}  [Электронный ресурс]\,{}: сайт. URL:\url{http://younglinux.info/sites/default/files/python_structured_programming.pdf}. (дата обращения: 11.01.2015).
\bibitem{pythonbook}М.~Лутц.  \href{https://vk.com/doc10903696_196246835?hash=4b80f3cf914c7d65dd&dl=f90686bee18e565271}{Изучаем Python, 4-е издание.}-- Пер.~с англ.--СПб.:Символ-Плюс, 2011. 1280~с.,~ил.
\bibitem{nominatim} OpenStreetMap Nominatim: Search.  [Электронный ресурс]\,{}: сайт. URL:\url{http://wiki.openstreetmap.org/wiki/Nominatim}. (дата обращения: 11.01.2015).
\bibitem{sethi}S.~P.~Sethi, G.~L.~Thomson. Optimal Control Theory: Applications to Management Science and Economics. 2nd Edition. 2005. 506~pp.
\end{thebibliography}

\chapter*{Приложение}
\label{lastpage}
\end{document}

%%%%%%%%%%%%%%%%%%%%%%%%%%%%%%%%%%%%%%%%%%%%%%%%%%%

% Local Variables:
% TeX-parse-self: t
% TeX-auto-save: t
% TeX-master: t
% End:
